\section {History of Ecological Thinking}

The credit for coining the word "ökologie" is generally given to Ernst Haeckel in an 1866 publication where he suggested that this new term should refer to the broad study of "nature's economy" \cite{worster_1977}. This is not to suggest that ecological thinking only has a history of 160 years. If we consider a broad definition of ecology as "... the study of the structure and function of nature, it being understood that mankind is a part of nature" we can trace the lineage of ecological thought back to classical Greece \cite{odum_1953}. Indeed, if we look to the history of thought on nature and the environment more broadly, we find that humans have grappled with three general nature based questions \cite{glacken_1967}: \begin{itemize} \item What is nature's influence on man? \item What is man's influence on nature? \item Is there a grand purpose in these relationships? \end{itemize} Stated in a more ecological sense, these questions can be reduced to the following: \begin{itemize} \item What is the influence of the non-living environment on living organisms? \item What is the influence of living organisms on the non-living environment? \end{itemize} For most of history we have struggled with the first question. With the recent turn to the Anthropocene we have engageed more deeply with the second question. The third question is best left to the theoligans. In any case, there is a long history of thought on the human relationship with both the living and the non-living environment in which we find ourselves.


\subsection{organismic vs mechanistic views of the world}

\cite{ranganathan_1931}
\cite{tansley_1935}

\subsection{holostic vs reductionistic approaches to Nature and Ecology}

\cite{holling_1998}
\cite{clements_1936}
\cite{gleason_1939}

\subsection{Nature as stable vs Nature as chaotic}

\cite{odum_1953} 
\cite{holling_1973}
\cite{barbour_1996}

\subsection{Ecology as science vs Ecology as social movement}

Ecologists as activist filled with recomendations for what "should" be conserved or saved similar to Librarians as activists deciding on what "should" be curated and preserved
