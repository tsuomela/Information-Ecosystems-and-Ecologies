\section {History of Ecological Thinking}

The credit for coining the word "ökologie" is generally given to Ernst Haeckel in an 1866 publication where he suggested that this new term should refer to the broad study of "nature's economy" \cite{worster_1977}. This is not to suggest that ecological thinking only has a history of 160 years. If we consider a broad definition of ecology as "... the study of the structure and function of nature, it being understood that mankind is a part of nature" we can trace the lineage of ecological thought back to classical Greece \cite{odum_1953}. Indeed, if we look to the history of thought on nature and the environment more broadly, we find that humans have grappled with three general nature based questions \cite{glacken_1967}: \begin{itemize} \item What is nature's influence on man? \item What is man's influence on nature? \item Is there a grand purpose in these relationships? \end{itemize} Stated in a more ecological sense, these questions can be reduced to the following: \begin{itemize} \item What is the influence of the non-living environment on living organisms? \item What is the influence of living organisms on the non-living environment? \end{itemize} For most of history we have struggled with the first question. With the recent attention to the Anthropocene we have engaged more deeply with the second question. The third question is best left to the theologians. In any case, there is a long history of thought on the human relationship with both the living and the non-living environment in which sustains us.

In any case, it is not really the study of ecology with which we concern ourselves in this essay, instead the concept of the ecosystem is the object of inquiry.

\subsection{ecology and ecosystems}



\subsection{organismic vs mechanistic views of the world}

\cite{tansley_1935}

\subsection{Holistic vs reductionist approaches to Nature and Ecology}

outline of the debate \cite{holling_1998, worster_1977}. \begin{itemize} \item the holistic view, ecosystems as a whole greater than the sum of its parts-holism \cite{clements_1936} \item the individualistic view, a call to focus on the components of the system which is knowable \cite{gleason_1939} \end{itemize}

Holling makes a particularly insightful observation that the analytical approach (reductionist) tends to come up with "exactly the right answer to the wrong question" while the integrative approach (holistic) asks "exactly the right question but [produces a] useless answer" \cite[][p. 3]{holling_1998] 

Final notes: This debate is far from over in ecology as a research program and indeed it is articulated with the same debate in the academy writ large. This is one area where the "information ecosystem" metaphor is very useful, there is a constant tension between epistemological approaches of understanding how  parts function and understanding how wholes function. Perhaps this tension is like (a simile now) the competing wave and particle theories of light in physics. Both theories represent valid ways of interpreting the world and the choice of theory depends on the final intentions or project of the researcher.

\subsection{Nature as stable vs Nature as chaotic}

stable \cite{odum_1953} 
resistance and resilience of systems \cite{holling_1973}
turn to the unknowable\cite{barbour_1996}

\subsection{Ecology as science vs Ecology as social movement}

Ecologists as activist filled with recommendations for what "should" be conserved or saved similar to Librarians as activists deciding on what "should" be curated and preserved
