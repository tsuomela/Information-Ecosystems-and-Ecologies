\section {Background: Ecology and Ecosystems}

The credit for coining the word "ökologie" is generally given to Ernst Haeckel in an 1866 publication where he suggested that this new term should refer to the broad study of nature's economy which he likely drew from Linnaeus’ 18th century conception of the “economy of nature” \cite{worster_1977}. This does not suggest that ecological thinking only has a history of 160 years. Instead, if we consider a broad definition of ecology as "... the study of the structure and function of nature, it being understood that mankind is a part of nature" we can trace the lineage of ecological thought back to classical Greece \cite{odum_1953}. Indeed, if we look to the history of thought on humans and their relationships with nature and the environment more broadly, we find three general questions \cite{glacken_1967}: \begin{itemize} \item What is nature's influence on man? \item What is man's influence on nature? \item Is there a grand purpose in these relationships? \end{itemize} Stated in a more ecological sense, these questions can be reduced to the following: \begin{itemize} \item What is the influence of the non-living environment on living organisms? \item What is the influence of living organisms on the non-living environment? \end{itemize} For most of history we have struggled with the first question and only with the thinking of Darwin did we begin to take it seriously that non-human organisms could be influenced by their surrounding environment. Then with the recent attention to the Anthropocene we have engaged more deeply with the second question and realized that human and non-human life alters the environment at planetary scales. The importance of the third question from Glacken is diminishing in academic circles and is generally left to the theologians, yet holistic approaches may be just as important to understanding nature as the concepts of stocks and flows of material resources borrowed from economic thought. But we digress.  

There are several different forms of ecology somewhat based on scale: individual, population, community, ecosystem 

Yet it is not really the discipline of ecology with which we concern ourselves, instead the concept of the ecosystem is the object of inquiry. Perhaps one of the problems with the information ecosystem metaphor is the confusing of ecology with ecosystems [as we shall see later]. In any case an ecosystem is understood as a community of living organisms–plants, animals, fungi, and so on–and the set of relationships between themselves and with their surrounding non-living environment \cite{tansley_1935, odum_1953}. Communities and relationships – ecosystems. The term ecosystem was introduced less than 100 years ago in 1935 by Arthur Tansley, an ecologist frustrated with the use of organismic metaphors to describe natural communities and their environs. He was also frustrated with organismic metaphors applied to “human communities as habitually so constituted” and wanted to clarify the notion that the ‘ecosystem’ is a human abstraction of climate, earth, and life that does not exist outside of human thought \cite{tansley_1935}.

Since this beginning the concept of the ecosystem has incorporated the notion of natures economy from early ecologists and developed a set of more refined approaches to thinking about this abstraction. The list is far too long to include all, but several key points are revelant to this essay. Evolution, competition, and cooperation as conceived by Darwin apply also to the ecosystem: In early thinking the ecosystem progressed through several stages to a reach a final stable state in which it was mature. Currently this concept of an evolved ecosystem is debunked and instead we think of dynamic equilibriums to which ecosystems tend to gravitate towards, yet with sufficient disturbance the system may gravitate to more than one stable point (similar to strange atractors in chaos theory). This approach drew heavily from cybernetic systems theory, the notion of the self-governing system with a complex of negative and positive feedback loops that led to emergent properties of the system in question. Directly related to this system based thinking are notions of sustainability and the role of competition, cooperation, co-dependence, synergies, limiting resources, and keystone species (the last of which, if removed the ecosystem spirals to a different dynamic equilibrium). As mentioned in the introduction, all of these concepts are somewhere buried in the metaphor of information ecosystems.

\subsection{ecology vs ecosystems}

The term "ecosystem" was introduced in 1935 by Arthur Tansley, an ecologist frustrated with the use of organismic metaphors to describe natural communities and their environs \cite{tansley_1935}. 

To this day the ecosystem is generally understood as a community of living organisms--plants, animals, fungi, and so on--and the set of relations with their surrounding non-living environment \cite{tansley_1935, odum_1953}. 

Perhaps one of the problems with the information ecosystem metaphor is the confusing of ecologies with ecosystems [NO definitely make this mistake]



\subsection{Evolution, competition, and cooperation}

A brief discussion of evolution, co-dependence, and co-evolution. This must include notes on population ecology - competition, resources, carrying capacity, etc.

\subsection{organismic vs mechanistic views of the world}

A brief intro to this idea?

\subsection{Holistic vs reductionist approaches to Nature and Ecology}

outline of the debate \cite{holling_1998, worster_1977}. \begin{itemize} \item the holistic view, ecosystems as a whole greater than the sum of its parts-holism \cite{clements_1936} \item the individualistic view, a call to focus on the components of the system which is knowable \cite{gleason_1939} \end{itemize}

Holling makes a particularly insightful observation that the analytical approach (reductionist) tends to come up with "exactly the right answer to the wrong question" while the integrative approach (holistic) asks "exactly the right question but [produces a] useless answer" \cite[][p. 3]{holling_1998} 

Final notes: This debate is far from over in ecology as a research program and indeed it is articulated with the same debate in the academy writ large. This is one area where the "information ecosystem" metaphor is very useful, there is a constant tension between epistemological approaches of understanding how  parts function and understanding how wholes function. Perhaps this tension is like (a simile now) the competing wave and particle theories of light in physics. Both theories represent valid ways of interpreting the world and the choice of theory depends on the final intentions or project of the researcher.

\subsection{Nature as stable vs Nature as chaotic}

- stable \cite{odum_1953} 
- resistance and resilience of systems \cite{holling_1973}
- turn to the unknowable\cite{barbour_1996}

\subsection{Diversity and stability}

- diversity measures in ecology and the relation to information theory (shannon/weaver). This is when ecology became a 'science'
- diversity's relationship to stability (as theorized by many)
- the keystone species' role in stability

\subsection{Ecology as science vs Ecology as social movement}

Ecologists as activist filled with recommendations for what "should" be conserved or saved similar to Librarians as activists deciding on what "should" be curated and preserved
