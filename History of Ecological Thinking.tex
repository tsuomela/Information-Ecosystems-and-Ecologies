\section {Background: Ecology and Ecosystems}

The credit for coining the word "ökologie" is generally given to Ernst Haeckel through his 1866 publication where he suggested that this new term should refer to the broad study of nature's economy. He likely drew from Linnaeus’ 18th century conception of the “economy of nature” \cite{worster_1977}. This recent coinage does not suggest that ecological thinking only has a history of 160 years. The Latin roots of the word ecology literally translate to the "study of the house" and if we consider a modern definition of ecology as "... the study of the structure and function of nature, it being understood that mankind is a part of nature" we can trace the lineage of ecological thought back to classical Greece \cite{odum_1953}. Indeed, if we look to the history of thought on humans and their relationships with nature more broadly, we find the persistence of three general questions \cite{glacken_1967}: \begin{itemize} \item What is nature's influence on man? \item What is man's influence on nature? \item Is there a grand purpose in these relationships? \end{itemize} Stated in a more ecological sense, these questions can be rephrased: \begin{itemize} \item What is the influence of the non-living environment on living organisms? \item What is the influence of living organisms on the non-living environment? \end{itemize} For most of history we have struggled with the first question. Only with the thinking of Darwin did we begin to take it seriously that non-human organisms could be influenced by their surrounding environment. Then with the recent attention to the Anthropocene we have engaged more deeply with the second question. While it is easy to observe relatively small-scale human influences on the environment, we now understand that both human and non-human life alters the environment at planetary scales. The importance of the third question from Glacken is diminishing in academic circles and is generally left to the theologians, yet holistic approaches may be just as important to understanding nature as the concepts of stocks and flows of material resources borrowed from economic thought. 

But we digress. It is not really the discipline of ecology with which we concern ourselves, instead the concept of the ecosystem is the object of inquiry. The term ecosystem was introduced less than 100 years ago in a 1935 paper written by Arthur Tansley, an ecologist frustrated with the use of organismic metaphors to describe biotic communities and their environs. This frustration also extended to applying the metaphor to “human communities as habitually so constituted.” His goal in coining the term was to clarify the notion that the "ecosystem" is an \textit{abstraction} of climate, earth, and life that does not exist outside of human thought \cite{tansley_1935}. An ecosystem is understood as a community of living organisms--the \textit{biocoenosis}: plants, animals, fungi, and so on--and the set of relationships between themselves and with their surrounding non-living environment--the \textit{biotope} \cite{tansley_1935, odum_1953}. Communities, evirons, and relationships--ecosystems. According to the Oxford English Dictionary, the term ecology has recently come to refer to simply the relationships found in an ecosystem as well \cite{oed_2008}. As noted below, one of the difficulties with the information ecosystem metaphor is the confusion of ecologies with ecosystems. For purposes of clarity in this essay the term ecology(ies) will be used to refer to the study(ies) of ecosystems and linked socio-natural systems--unless explicitly noted otherwise--not the set of relations found within an ecosystem. 

As the nascent field of ecology entered into the post WWII quantitative revolution of the 1950s and 1960s, ecologists struggled to claim an identity as data driven scienctists. Much like other disciplines at this time, ecology struggled with the continuing philosophical debates around holism vs. reductionism and organismic vs. mechanistic explanations of nature \cite{barbour_1996}. A key proponent of the holistic approach was Frederick Clements who claimed that ecosystems were wholes that were greater than the sums of their parts. He also claimed that entire ecosystems "evolved" along an observable succession of steps to maturity. Thus a particular combination of soil, climate and organisms will always proceed to a defined stable endpoint \cite{clements_1936}. His principal adversary was Henry Gleason who promoted the view that there was no ecosystem, clearly echoing the ideas of Tansley, and that all "plant associations" (his term for plant communities or ecosystems) are made of \textit{individuals} competing for resources and space. He paid particular attention to the problem of scale in ecology and how choice of the scale of analysis will define how an observing scientist will classify an association of plants \cite{gleason_1939}. The view of Clements led to a (non-quantifiable) holisitic approach and those of Gelason led to a (quantifiable) reductionist approach. This debate is still active, but due to the preference for quantitative work in scientific funding agencies, it is the reductionist approach that moves forward faster. As prominent ecologist CS Holling insightfully observes, the analytical [reductionsit] approach tends to come up with "exactly the right answer to the wrong question" while the integrative [holistic] approach asks "exactly the right question but [produces a] useless answer" \cite[][p. 3]{holling_1998}.

This debate is far from over in ecology as a research program and indeed it is articulated with the same debate in the academy writ large. This is one area where the "information ecosystem" metaphor is very useful, there is a constant tension between epistemological approaches of understanding how  parts function and understanding how wholes function. Perhaps this tension is like (a simile now) the competing wave and particle theories of light in physics. Both theories represent valid ways of interpreting the world and the choice of theory depends on the final intentions or project of the researcher. This approach is similar to that of the roman god Janus, always looking in two directions: forwards or backwards (in time), up or down, inwards or outwards, and so on. What he sees is a matter of the chosen perspective. The phsycologist Arthur Koestler's concept of the "holon" is also useful to resolve this tension. A holon is a part-whole that can be conceptualized (abstracted) at many different scales within biological, biogeochemical, and social systems that has both self-assertive and integrative tendancies: "the self-assertive tendancy is the dynamic expression of the holon's wholeness, the integrative tendancy, the dynamic expression of its partness" \cite[][p. 56] {koestler_1967}.

Parallel to this reductionist-integrative debate in ecology, and still in the context of the post WWI quantitative revolution, was the emergence of information and communication theory led by the work at Bell Labs and other growing technology firms. With the advent of digital computers during the war, and the subsequent interest in \textit{securley} managing larger and larger sets of information, people began to speak of and study the "information environment." One of the key problems that was addressed in this work was the role of information diversity in a data stream; this was particularly important on noisy channels for the transmission of data on metal wires (think telephone and telegraph lines). A major advance was made by the work of Claude Shannon at Bell Labs who proposed a mathematical theory of information diversity on noisy channels \cite{shannon_1948}. While this text is still considered cannonical in computer and information science, ecologists also picked it up and started to use the same methods for calculating diversity of food webs or \textit{trophic diversity} \cite{macarthur_1955} and then for the diversity of organisms in an ecosystem \cite{margalef_1957}. Margalef was "fully conscious ... of the risk of displeasing both mathematicians and biologists" with the application of diversity models from information theory to ecosystem biodiversity and was also aware that this mathematical model for biodiversity was far from perfect \cite{margalef_1957}. Shannon's diversity index is still taught to students of ecology as part of a standard set of methodological tools, nevertheless. This is an important early and direct direct link between information science and ecology.

During this time several other key debates in ecology also matured, such as those around competition vs. cooperation in evolutionary processes, the role of biological diversity in the stability of ecosystems, and more broadly whether nature is stable or chaotic. Evolution, competition, and cooperation as conceived by Darwin have been applied to ecosystem thinking since its inception, although not without difficulties and ideological differences. As originally conceived evolution was largely competitive; individuals from certain species were in constant struggle to claim and maintain their places in nature's economy. His theory of "descent with modification" was developed in the context of Malthusian thinking on human populations in which food resources increase in a linear fashion and population increases in a exponential manner; quickly survival becomes a matter of "red in tooth and claw" not only between species, but within the same species \cite{stoddart_1966}. Well know to history is that this idea of comepetive evolution was quickly applied to human societies and "the survival of the fittest" was applied to both the relations between humans and between humans and the environment. The first with dire consequences as shown by Hitler's adoption of the notion of textit{lebensraum}--in which political states are said to "evolve" and claim spaces in the global economy--to justify his crusade to exterminate the Jews, and the second in much managerial rhetoric in which it is our destiny as humans to control and dominate nature \cite{stoddart_1966,worster_1977}.  

As mentioned above, in early thinking the ecosystem evolved through several stages to a reach a final stable state in which it was mature. This concept of an "evolved" ecosystem is now debunked and instead we think of "dynamic equilibriums" to which ecosystems tend to gravitate towards. Crafword Holling's twin concepts of systemic resistance to change and the resilience of a system after a major disturbance are often used to explain dynamic equilibriums in ecosystems. When observing populations of organisms that either compete for knowable quantities of resources (generally plant communities) or are in predator-prey relationships (generally animal communities) the demographics tend to hover around a stable point unless there is a disturbance of some sort (fire, weather event, human intervention, and so on) \cite{holling_1973}. One year after Holling's publication, the ecosystem modeler--then a very new field that used computers to model species populations in a study ecosytem--Robert May questioned the idea of stable points in population dynamics and introduce the possiblity of using chaotic models to explain observed population phenomena in ecosystems. With sufficient disturbance the system may gravitate to more than one stable point similar to strange attractors in chaos theory, or the system may bifurcate and enter periods of instability \cite{may_1974}. This approach drew heavily from cybernetic systems theory, the notion of the \textit{self-governing} system with a complex of negative and positive feedback loops that led to emergent properties of the system in question. Inevitably this led to complexity and finally back to holistic approaches to explaining ecosystems as unkowable as a whole \cite{barbour_1996}.

Directly related to this kind of system based thinking are notions of stability and sustainability. The role of competition, cooperation, co-dependence, and synergistic relationships can be clearly articulated with the sustainability of a given ecosystem. Also important to sustainability of the ecosystem are concepts of limiting resources and how a certain material resource such as Nitrogen may govern the behaviour of the system, and the concept of the keystone species which, if removed from the living community may cause the ecosystem to spiral towards a different dynamic equilibrium. Even after the idea of the evolved ecosystem was laid to rest, many ecologist still explained ecosystems as "naturally" stable \cite{odum_1953}. Through the use of the mathematical diversity models from Shannon

 Notwithstanding this relatively short history of "ecology" and "the ecosystem," both of the terms have become important parts of the English language and are considered household terms as well as referents to forms of scientific inquiry and socio-natural systems. Not only have ecologists and ecosystems thinking contributed knowledge to our understandings of natural processes, but several ecologists/biologists are credited with initiatiating the environmental movement of the 1970s: the writings of Aldo Leopold, Rachel Carson, and Paul Erlich are often cited as influential in the beginning of this movement. In a sense these reseachers were powerful contributors to the idea of the academic activist. As mentioned in the introduction, all of these concepts are somewhere buried in the metaphors of information ecosystems, publishing ecosystems, and research ecosystems.


