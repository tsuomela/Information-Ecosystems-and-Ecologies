\section{Critique of Information Ecology}

\subsection{Cycles and Flows:}

Equivalence between information and energy. From the thermodynamic interpretation of Shannon to later work by Dretske and others moving information into a primary ontological function.

Also equivalence between moeny (information ecosystems) and energy (real ecosystems)

\subsection{The attempt to bring humans back into the technology conversation (good!):}

Nardi and ODay place a great deal of emphasis on the local viewpoint opened up by the concept of information ecology. An explicit motivation for this local focus is to offer an alternative to the system critique of technology, which Nardi and ODay believe ignores the ways in which people may intervene in technological systems.

NO were writing in the 1990s before the internet and world wide web became one of the dominant information forces in the world. But they were also writing during the consolidation of the neoliberal Washington consensus in the economic and political realms. In combination I think the growth of neoliberalism and the internet have led to a networked society. Manuel Castells is one of the major sociologists working on this topic. In the networked society local accommodations can quickly be coopted by larger systemic forces. The argument is that localism is a difficult location from which to build a technical critique of the networked society. Should we be pursuing other avenues? Other sources than Castells could be used to build this argument.

\subsection{How is Diversity understood?}

Another important goal for the IE program proposed by NO is to value diversity in information ecologies. DP also discuss the value of diversity for business organizations. But does the idea of diversity really align with the discussion of diversity that occurs in ecology?

\subsection{Greening of the information industry:}

I also believe that a critique could be made of NO and DP that sees the adoption of ecological metaphors as a bit of grandstanding on their part. Identifying with ecology may have a political and rhetorical benefit. Who would want to complain about ecology, especially when the implicit opposite for both NO and DP is a type of technocratic managerialism.

\subsection{Are the emergent properties Natural or set by humans?}

Much of what happens in the information environment is conditioned by standards which are mostly developed by institutions and businesses. Is there really room for indivduals to intervene at a local level on these standards? And once they are adopted path dependence may set in, making change very difficult.

Another interesting concept to consider in IE is the idea of emergence. Does linking emergence to biological and environmental concepts beneficial [this has been done for a long time, see both Glacken, Worster, Lovelock, etc - hotly debated subject]? These concepts are also present in complexity and chaos studies which tend to have roots in more scientific fields such as physics, and computer science [cybernetics, self governing systems, homeostasis].

\subsection{Are human created phenomena natural}

Within certain critical academic circles naturalizing a phenomena that is a human creation—somehow perceiving the social phenomena as natural—is considered dangerous. A good example might be an explanation of the conquest of Peru as inevitable and ‘natural’ because of superior European weaponry and the disease trajectory from Europe to South America (to simplify a famous argument). Nowhere in this explanation is it mentioned that the conquistadores were social outcasts with a tendency for violence and disrespect of life; perhaps another factor in the conquest. 

\subsection{Where is Nature in the metaphor?}

Our admittedly incomplete review of the literature suggests that the use of the term information ecosystem either as a metaphor or as a simple vernacular reference actually renders nature more incivisble invisible instead of more visible. Is it possible that the use of the term information ecosystem further disconnects us from our natural environment?  

\subsection{Human Communities are NOT ecosystems, lack of morality}

The solace that I found with my discomfort for the ecological metaphor in information systems came from the words of Richard Stallman, who launched the free software movement. “It is inadvisable to describe the free software community, or any human community, as an “ecosystem,” because that word implies the absence of ethical judgment.”  Stallman continues:
 
\quotation{The term “ecosystem” implicitly suggests an attitude of nonjudgmental observation: don't ask how [sic] what should happen, just study and understand what does happen. In an ecosystem, some organisms consume other organisms. In ecology, we do not ask whether it is right for an owl to eat a mouse or for a mouse to eat a seed, we only observe that they do so. Species' populations grow or shrink according to the conditions; this is neither right nor wrong, merely an ecological phenomenon, even if it goes so far as the extinction of a species.
By contrast, beings that adopt an ethical stance towards their surroundings can decide to preserve things that, without their intervention, might vanish—such as civil society, democracy, human rights, peace, public health, a stable climate, clean air and water, endangered species, traditional arts…and computer users' freedom.}

Perhaps these words also apply to information systems—as these systems are indeed social phenomena—and although the ecosystem metaphor may be convenient and perhaps even practical, we should be careful with expressing an implicit lack of morality in the world of information.
