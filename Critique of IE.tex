\section{Critique of Information Ecology}

Equivalence between information and energy. From the thermodynamic interpretation of Shannon to later work by Dretske and others moving information into a primary ontological function.


My initial critiques of information ecology as developed so far.

Nardi and ODay place a great deal of emphasis on the local viewpoint opened up by the concept of information ecology. An explicit motivation for this local focus is to offer an alternative to the system critique of technology, which Nardi and ODay believe ignores the ways in which people may intervene in technological systems.

NO were writing in the 1990s before the internet and world wide web became one of the dominant information forces in the world. But they were also writing during the consolidation of the neoliberal Washington consensus in the economic and political realms. In combination I think the growth of neoliberalism and the internet have led to a networked society. Manuel Castells is one of the major sociologists working on this topic. In the networked society local accommodations can quickly be coopted by larger systemic forces. The argument is that localism is a difficult location from which to build a technical critique of the networked society. Should we be pursuing other avenues? Other sources than Castells could be used to build this argument.

Another important goal for the IE program proposed by NO is to value diversity in information ecologies. DP also discuss the value of diversity for business organizations. But does the idea of diversity really align with the discussion of diversity that occurs in ecology?

I also believe that a critique could be made of NO and DP that sees the adoption of ecological metaphors as a bit of grandstanding on their part. Identifying with ecology may have a political and rhetorical benefit. Who would want to complain about ecology, especially when the implicit opposite for both NO and DP is a type of technocratic managerialism.

Much of what happens in the information environment is conditioned by standards which are mostly developed by institutions and businesses. Is there really room for indivduals to intervene at a local level on these standards? And once they are adopted path dependence may set in, making change very difficult.