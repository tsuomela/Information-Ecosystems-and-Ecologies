\section{Adjacent Ideas and Ecological Adoptions}

I want to use this section of the paper to discuss some of the other disciplines that have taken up ecological ideas or themes. My hope is that this discussion may help show how information science can benefit from ecological thinking, and to possibly suggest improvements in the application of ecological ideas to the field of information science.

Ecological concepts have been adopted by many different disciplines over the course of the twentieth century. An examination of some of these adoptions may help to illuminate the adoption of those ideas into information studies in the late 1990s. Three examples of ecological concepts migrating into other disciplines will be discussed in this section: cognitive ecology, media ecology, and political ecology. Each of them shows how ecological concepts can be adopted by a discipline for many different purposes. Purposes which may be renegotiated at any time.

In a review of cognitive ecology Hutchins \cite{hutchins_cognitive_2010} discusses the sources of the field and how it interacts with cognitive science. He writes that "Cognitive ecology is the study of cognitive phenomena in context. Elements of cognitive ecology have been present in various corners, but not the core, of cognitive science since the birth of the field. It is now being rediscovered as cognitive science shifts from viewing cognition as a logical process to seeing it as a biological phenomenon." (705) In the abstract he highlights how cognitive ecology "points to the web of mutual dependence among elements of a cognitive ecosystem."

A number of interesting rhetorical and thematic connections are being made in these statements. Hutchins contrasts logical and biological approaches to understanding cognition. Later in the paper he describes how early cognitive science was faced with a tension between reductionism and holism. Two schools of thought, a cybernetic and an information processing approach, emerged from the early ferment of the field. Cyberneticists, such as Gregory Bateson, emphasized the interactions between mind and the environment. Information processing advocates concentrated on the parallels between the digital computer and the mind. The focus on the digital computer reduced the activity of the mind to symbolic event processing, relegating perceptual systems and the motor systems used to interact with the world to the periphery of the field of cognitive science. According to Hutchins, cognitive science is revising itself and reexamining the connections between the world and the mind. Cognitive ecology is an example of how these issues are being addressed.

Bateson wrote an influential book in 1972 titled \textit{Steps to an Ecology of Mind}. The book was a theoretical manifesto for paying greater attention to the interconnections between environment and the mind. Other research programs emerged during the 1970s to form the base for cognitive ecology. Ecological psychology stressed the coupling between organism and environment, while cultural-historical activity theory stressed the internalization of interpsychological processes during childhood development. Recent developments linking mind and environment have coalesced under the umbrella terms of embodied cognition and enaction.

One of the themes emerging from the development of cognitive ecology is the contrast between the digital and the biological. Cognitive science, as a field, chose to use a digital-technological paradigm in order to manage the scope of its studies and to define the appropriate units of analysis for its research efforts. Critics of this approach were present from the beginning, but often marginalized. A similar story is told by DP and NO in their narratives of information ecology, both of which employ the theme of moving away from a narrow technical focus on the systems for information management, which stand in for technological development, especially in the form of engineering with all of the symbolic baggage that term implies, toward a larger, contextually-dependent, vision of what information entails. Just as the cognitive ecologists wish to move away from viewing cognition as just another form of information processing, the information ecologists want to view the organization as something greater than a digital computer. Context becomes one of the key concepts for both motivating and operationalizing this transformation. Motivating because who would object to incorporating the context into a proper discussion of an information environment, and operationalizing because the idea of context points to a new unit of analysis beyond the individual software deployment toward the people and groups which surround and interact with any sociotechnical system.

Another theme within Hutchins description of cognitive ecology is the idea of mutual dependence. NO use mutual dependence as one of their key themes within information ecology, and the interconnected organizations which surround a business are suggestive of the arguments made by DP. In \textit{Cognition in the Wild} Hutchins described the process used to navigate large Navy ships, which are excellent examples of the mutual dependence between people, technology, and information. Each component of the system, or ecology, needs to be functioning well in order to achieve the overall goal of determining the ship's position. Mutual dependence is also a valuable feature within discourses about the value of biodiversity.


Media ecology was first used as a term by Neil Postman in the early 1970s in order to turn attention to the environment in which media and people interact. Postman was influenced by Marshall McLuhan, Walter Ong, and Harold Innis who were all involved in the study of communication and media during the 1950s and 1960s. These thinkers were often reacting to the heavily quantitative research traditions popular in communication studies after World War Two. Turning toward the environment in which communication and media interacted suggested a way to broaden the analysis of media in order to develop a broader critique of media in society. One definition of media ecology is as

\begin{quote}
how the form and inherent biases of communication media help create the environment or symbolic and cognitive structure in which people symbolically construct the world they come to know and understand, as well as its social, economic, political, and cultural consequences. \cite{lum_introduction:_2000}
\end{quote}

One root of media ecology may be traced to the works of Lewis Mumford who wrote multiple critiques of technology and its development in modern society. He introduced the idea of the mega-machine which was distorting society. This critique of technology parallels the work by Jacques Ellul and Langdon Winner which were used explicitly by NO to motivate the development of information ecology.