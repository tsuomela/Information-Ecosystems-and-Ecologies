\section{Adjacent Ideas and Ecological Adoptions}

I want to use this section of the paper to discuss some of the other disciplines that have taken up ecological ideas or themes. My hope is that this discussion may help show how information science can benefit from ecological thinking, and to possibly suggest improvements in the application of ecological ideas to the field of information science.

Ecological concepts have been adopted by many different disciplines over the course of the twentieth century. An examination of some of these adoptions may help to illuminate the adoption of those ideas into information studies in the late 1990s. Three examples of ecological concepts migrating into other disciplines will be discussed in this section: cognitive ecology, media ecology, and political ecology. Each of them shows how ecological concepts can be adopted by a discipline for many different purposes. Purposes which may be renegotiated at any time.

Cognitive ecology is described by Hutchins 

Media ecology was first used as a term by Neil Postman in the early 1970s in order to turn attention to the environment in which media and people interact. Postman was influenced by Marshall McLuhan, Walter Ong, and Harold Innis who were all involved in the study of communication and media during the 1950s and 1960s. These thinkers were often reacting to the heavily quantitative research traditions popular in communication studies after World War Two. Turning toward the environment in which communication and media interacted suggested a way to broaden the analysis of media in order to develop a broader critique of media in society. One definition of media ecology is as

\begin{quote}
how the form and inherent biases of communication media help create the environment or symbolic and cognitive structure in which people symbolically construct the world they come to know and understand, as well as its social, economic, political, and cultural consequences. \cite{lum_introduction:_2000}
\end{quote}

One root of media ecology may be traced to the works of Lewis Mumford who wrote multiple critiques of technology and its development in modern society. He introduced the idea of the mega-machine which was distorting society. This critique of technology parallels the work by Jacques Ellul and Langdon Winner which were used explicitly by NO to motivate the development of information ecology.