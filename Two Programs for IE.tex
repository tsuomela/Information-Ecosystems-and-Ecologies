\section{Two Programs for Information Eco(systems-ologies)}

As the "information ecosystem" metaphor initiated in academic circles moved into popular business contexts in the 1990s, two identifyable programs were developed. The first is a managerial approach outlined as what an ecologist or a natural resource manager might call adaptive management [needs cite]. Davenport and Prusak suggested an "ecological" approach to information management that explicitly acknowledges the human-technology nexus and cycles of achievement and constant re-evaluation of objectives that are moving targets. Their effort intends to improve data managment in the world of corporate business. The second program also contains a plea to re-incorporate people into technological solutions for data management in the corporate world, but comes from the perspective of research labs and user interface design. Nardi and O'Day's work as an outcome of their time at the research labs of Apple computers and Xerox repsectively provides useful insights into [sic] "information ecologies," but it can also be seen as an attempt to green the information industry and make it look more human than it was at the time of writing [needs re-wording]. Both of these approaches share a human-machine system as organism metaphor, almost as though they are trying to understand how we have become cyborgs [!?!?!? - perhaps not, but a reference to Donna Harroway's work on cyborgs might be appropriate].

\subsection{Davenport Prusak}

Davenport and Prusak position their 1997 book, \textit{Information Ecology: mastering the information and knowledge environment}, as an antidote to what they term the "machine-engineering" view of technology adoption and use. From the beginning their intended audience is business managers who are involved in the regular operations of organizational information systems. In other words, the IT guys. Information ecology is an antidote to the engineering viewpoint because it "puts how people create, distribute, understand, and use information at its center" (ch1), instead of software design. It is an "ecological approach" to information management to counter technological approaches.

Information ecology is defined as "the science of understanding and managing whole environments" and they propose 4 basic beliefs for the information ecology approach. (ch1)
\begin{enumerate}
\item information is not easily stored on computers—and is not "data"
\item the more complex an information model, the less useful it will be
\item information can take on many meanings in an organization
\item technology is only one component of the information environment and often not the right way to create change.
\end{enumerate}

Davenport and Prusak are clear that describing a organization as an information ecology is a metaphor, but they think the value of the metaphor is as a counterweight to the machine-engineering point of view mentioned above. Indeed, they employ the metaphor to bring people back into the world of information technology. One of the principal theses of their book is that technology alone cannot be a solution to data management problems, but that human resources, abilities, and networks are essential to improve data management in a corporate business context. A second major thesis in their book is that adaptive management approaches are necessary to sustain information systems.

In chapter 3 they write of four key attributes for information ecology

\begin{enumerate}
\item integration of diverse types of information;
\item recognition of evolutionary change;
\item emphasis on observation and description;
\item and focus on people and information behavior.
\end{enumerate}
While the fully ecological approach to information would adopt all of these attributes, each is valuable in its own right. Adopting any of them will help organizations move in a more ecological direction—an important fact to keep in mind when information ecology seems like a daunting project."

In addition to describing the beliefs and attributes of their information ecology approach, Davenport and Prusak acknowledge that any organization is embedded in multiple information environments. The internal company or organizational environment is the main focus of their argument, but any organization is also connected to other organizations, such as business partners, customers, and suppliers, as well as being involved in a larger environment of the marketplace. The boundaries between these environments is often fuzzy and potentially quite porous. It is interesting to note that DP make no mention of the natural environment as an explicit context for an organization. This absence makes their overall program for adopting information ecologies less radical than it could potentially be.

The book is composed of multiple case studies, as is typical in the business management literature. Chapters in the book discuss the topic of strategy, politics, culture, staff, management, and architecture. Some of the cases are more intriguing than others. One case summarizes the information ecology approach into 5 uses for or understandings of information in an organization: truth, guidance, scarcity, accessibility, and weight. The only one of these which may be mapped easily onto biological conceptions of ecology is scarcity. Mapping the others to biological concepts would be a challenge.

\subsection{Nardi O'Day}

Nardi and O'Day share some of the same concerns as DP about the impact of technology on human living, but they approach the problem from a different disciplinary perspective. The major disciplinary contributor to their analysis of information is anthropology, whereas DP were coming from a business management perspective.

The shared critique of technology is stated even more directly by NO. They describe a number of different common perspectives on technology and society. The first is the tool view of technology which is very similar to the machine-engineering perspective mentioned by DP. The tool viewpoint sees information technologies as just another item in the world of organizations which is used to accomplish tasks. The tool can be improved through better design processes but is itself morally neutral. The second view is of technology as a text which can be interpreted by scholars in order to discern the intentions and meaning of using a particular technology. The third view, and the one to which they devote most of their discussion, is the systems view.

The systems view is the strongest critique of technology. Propounded by people such as Jacques Ellul and Langdon Winner, the systems view argues that human lives are conditioned, perhaps even determined, by the technological world they inhabit. Winner, in particular, critiques three common myths about technology: that it is neutral, that it is controlled by people, and that people understand the technology they use. The fact that many technologies lead to unintended consequences is prima facie evidence for the last two myths.

Like DP, NO list a number of key ideas from ecology which they wish to apply to information.

\begin{enumerate}
\item Systemic relationships between the parts of an ecology.
\item Diversity of people and tools in an information system.
\item Co-evolution between the people and the tools in the information ecosystem.
\item Keystone species, or mediators who "build bridges across institutional boundaries and translate across disciplines" (53)
\item Locality, or the name (meaning) and habitation (location in the network/system) of a technology.
\end{enumerate}

The key point made by NO is that, although they are sympathetic to the systems critique of technology, they believe this critique is too totalizing and thus a deterrent to intervention. This idea is encapsulated in the slogan "rhetoric of inevitability" which they argue is too often part of the discussions around the effects of technology. Ecology offers a better metaphor because it is potentially more diverse than a community, draws out the importance of continual evolution, and introduces an urgency to technology discussions which can be directly linked to environmental crises. Instead of resistance to a total system, as described by Ellul or Winner, we should speak of engagement and participation, especially in the specific local situations in which each of us encounters any given technology.

Already we can begin to see a significant tension in the rhetoric of information ecologies. On the one hand we are being asked to pay attention to the urgency of global environmental and ecological breakdown, both biologically and informatically. But the places for engagement are framed as distinctly local. Granted we may indeed be more optimistic about the potential for local change but if global, or systemic, environmental challenges are really so urgent then resistance may be the more appropriate option.
