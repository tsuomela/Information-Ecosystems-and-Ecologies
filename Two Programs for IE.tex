\section{Two Programs for Information Ecologies}

As ecological thinking moved into popular business contexts in the 1990s, two identifiable programs of information ecology (as a set of relations) were developed. The first is a managerial approach outlined as what an ecologist or a natural resource manager might call adaptive management. Davenport and Prusak suggested an ``ecological'' approach to information management that explicitly acknowledges the human-technology nexus and cycles of achievement. This calls for a constant re-evaluation of objectives that are moving targets. Their effort intends to improve data management in the world of corporate business. 

The second program also contains a plea to re-incorporate people into technological solutions for data management in the corporate world, but comes from the perspective of research labs and user interface design. Nardi and O'Day's work is an outcome of their time at the research labs of Apple  and Xerox, and provides useful insights into ``information ecologies'' (ecologies as relationships, not as a discipline or research).

Both narratives of information ecology draw upon cognitive ecology and emphasize a moving away from a narrow technical focus on the systems for information management, which stand in for technological development, especially in the form of engineering with all of the symbolic baggage that term implies, toward a larger, contextually-dependent, vision of what information entails. Just as the cognitive ecologists wish to move away from viewing cognition as just another form of information processing, the information ecologists want to view the organization as something greater than a digital computer. 

\subsection{Davenport and Prusak}

Davenport and Prusak positions the 1997 book, \textit{Information Ecology: mastering the information and knowledge environment} as an antidote to what they term the ``machine-engineering'' view of technology adoption and use.  Information ecology is proposed as an antidote to the engineering viewpoint because it ``puts how people create, distribute, understand, and use information at its center'' (ch1), instead of focusing on software design as engineers are prone to do. The intended audience are business managers who are involved in the regular operations of organizational information systems; in other words, the IT guys. Davenport and Prusak argue that a holistic approach to management information systems is more effective than a systems- (in the sense of computer based system engineering) or engineering-based approach. The ``ecological approach'' behind this argument is simplistic because ecology is basically standing for any complex system with multiple parts and feedbacks between different levels, whether it is a natural or artificial system makes no difference.

They define information ecology as ``the science of understanding and managing whole environments'' and proposes 4 basic tenants for the information ecology approach. (ch1)
\begin{enumerate}
\item information is not easily stored on computers—and is not ``data''
\item the more complex an information model, the less useful it will be
\item information can take on many meanings in an organization
\item technology is only one component of the information environment and often not the right way to create change.
\end{enumerate}

The role of people in business organizations is particularly important for Davenport and Prusak.  They refuse to reduce information to data which can simply be transported around or stored by an organization using the typical tools of an IT engineer.  Instead, information can have multiple meanings, different levels of complexity, and not be dependent on any particular form of technology. By describing an organization as an information ecology Davenport and Prusak explicitly counter the machine-engineering point of view. The metaphor serves to bring people back into the world of information technology and supports one of the principal theses of the book : that technology alone cannot be a solution to data management problems. Human resources, abilities, and networks are essential for improving data management in a corporate business context. A second major thesis in their book is that adaptive management approaches are necessary to sustain information systems.

In chapter 3 four key components for information ecology are introduced: 

\begin{enumerate}
\item integration of diverse types of information;
\item recognition of evolutionary change;
\item emphasis on observation and description;
\item and focus on people and information behavior.
\end{enumerate}

The parallels between the key components of Davenport and Prusak and the ecological literature around adaptive management are quite clear. Both share an emphasis on incorporating human behavior into a system and the idea of iterative or evolutionary change.  In an ideal world, a business organization would adopt all of the key attributes in order to be more effective. But each attribute is also valuable in its own right, and adopting any of them will help organizations move in a more ecological direction.

In addition to describing the tenants and components of their information ecology approach, Davenport and Prusak acknowledge that any organization is embedded in multiple information environments. The internal company or organizational environment is the main focus of their argument, but any organization is also connected to other organizations, such as business partners, customers, and suppliers, as well as being involved in a larger environment of the marketplace. The boundaries between these environments is often fuzzy and potentially quite porous. It is interesting to note that there is no mention of the natural environment as an explicit context for an organization. The import of this omission is difficult to judge. On the one hand the book is aimed at business managers who are working in technology dominated information fields so discussing the natural environment might be considered out of scope. On the other hand they are directly borrowing terminology from the field of ecology which takes the natural environment to be its direct subject of study. The confusion between these two impulses is part of the reason why discussing information ecology is so difficult.

\subsection{Nardi O'Day}

Nardi and O'Day share some of the same concerns as Davenport and Prusak about the impact of technology on human living, but they approach the problem from a different disciplinary perspective. Given their background in anthropology the focus is not on the technology but the human cultures in which technologies are embedded. Previous work in this tradition grew out of science and technology studies, as well as human computer interaction research on the problems of ``user resistance'' in the adoption of new information systems \citep{star_1996}.

The critique of technology based solutions to information problems is stated even more directly by Nardi and O'Day. They describe three different common perspectives on technology and society. The first is the tool view of technology which is very similar to the machine-engineering perspective mentioned by Davenport and Prusak. The tool viewpoint sees information technologies as just another device used by organizations to accomplish tasks. Tools can be improved through better design and integration but are morally neutral qua technologies. The second view is of technology as a text which can be interpreted by scholars in order to discern the intentions and meaning of using a particular technology. The third view, and the one to which they devote most of their discussion, is the systems view.

The systems view is the strongest critique of technology based solutions to information problems. Propounded by people such as Jacques Ellul and Langdon Winner, the systems view argues that human lives are conditioned, perhaps even determined, by the technologies that surround them. Winner, in particular, critiques three common myths about technology: that it is neutral, that people have control over the technologies used by society, and that people understand the implications of the technology they use [is there a cite?]. The fact that many technologies lead to unintended consequences is prima facie evidence for the last two myths.

Nardi and O'Day list a number of key ideas from ecology which they wish to apply to information.

\begin{enumerate}
\item Systemic relationships between the parts of an ecology.
\item Diversity of people and tools in an information system.
\item Co-evolution between the people and the tools in the information ecosystem.
\item Keystone species, or mediators who ``build bridges across institutional boundaries and translate across disciplines'' (53)
\item Locality, or the name (meaning) and habitation (location in the network/system) of a technology.
\end{enumerate}

Of great significance to ``library ecosystems'' is their notion of keystone species. Indeed, they name corporate librarians at Apple and Xerox as the keystone species in their particular information ecologies (as sets of relationships). Certainly this makes librarians feel important, but it also raises questions as to whether people can really be a keystone species in any ecosystem.

The key point made by Nardi and O'Day is that, although they are sympathetic to the systems critique of technology, they believe this critique is too totalizing and thus a deterrent to intervention. This idea is encapsulated in the slogan ``rhetoric of inevitability'' which they argue is too often part of the discussions around the effects of technology. The idea of an information ecology is offered as an explicit alternative to the systems critique of information and technology. They contrast the term ecology to the term communities, arguing that the latter is too homogenous and the former implies greater diversity. Ecology also draws out the importance of continual evolution and change, an implication that is lacking for the term community.  Finally, information ecology introduces an urgency to technology discussions which can be directly linked to environmental crises.

An information ecology, according to Nardi and O'Day, offers an opportunity for local action which is not dependent on the larger, systemic critiques of technology.  Following a systemic critique of technology, as described by Ellul or Winner, is likely to lead to a sense of despair and powerlessness of individuals to affect the technologies used by society. Instead we should speak of engagement and participation, especially in the specific local situations in which each of us encounters any given technology. By acting where we have power, in the local habitations where the meaning of a technology is being created, we can influence the information ecologies around us. The rhetoric of ecology brings out the holistic relationships between people and tools, and points us to an arena where we can affect change.