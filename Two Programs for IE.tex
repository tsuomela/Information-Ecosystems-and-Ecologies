\section{Two Programs for Information Eco(systems-ologies)}

\subsection{Davenport Prusak}

Davenport and Prusak position their 1997 book, \textit{Information Ecology: mastering the information and knowledge environment}, as an antidote to what they term the "machine-engineering" view of technology adoption and use. From the beginning their intended audience is business managers who are involved in the regular operations of organizational information systems. In other words, the IT guys. Information ecology is an antidote to the engineering viewpoint because it "puts how people create, distribute, understand, and use information at its center" (ch1), instead of software design.

Information ecology is defined as "the science of understanding and managing whole environments" and they propose 4 basic beliefs for the information ecology approach. (ch1)
\begin{enumerate}
\item information is not easily stored on computers—and is not "data"
\item the more complex an information model, the less useful it will be
\item information can take on many meanings in an organization
\item technology is only one component of the information environment and often not the right way to create change.
\end{enumerate}

A more detailed comparison of these principles with those proposed by Nardi and O'Day will be discussed later is this paper.

Davenport and Prusak are clear that describing a organization as an information ecology is a metaphor, but they think the value of the metaphor is as a counterweight to the machine-engineering point of view mentioned above.

In chapter 3 they write of four key attributes for information ecology

\begin{enumerate}
\item "(1) integration of diverse types of information;
\item (2) recognition of evolutionary change;
\item (3) emphasis on observation and description;
\item and (4) focus on people and information behavior.
\end{enumerate}
While the fully ecological approach to information would adopt all of these attributes, each is valuable in its own right. Adopting any of them will help organizations move in a more ecological direction—an important fact to keep in mind when information ecology seems like a daunting project."



\subsection{Nardi O'Day}