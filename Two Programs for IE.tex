\section{Two Programs for Information Eco(systems-ologies)}

\subsection{Davenport Prusak}

Davenport and Prusak position their 1997 book, \textit{Information Ecology: mastering the information and knowledge environment}, as an antidote to what they term the "machine-engineering" view of technology adoption and use. From the beginning their intended audience is business managers who are involved in the regular operations of organizational information systems. In other words, the IT guys. Information ecology is an antidote to the engineering viewpoint because it "puts how people create, distribute, understand, and use information at its center" (ch1), instead of software design.

Information ecology is defined as "the science of understanding and managing whole environments" and they propose 4 basic beliefs for the information ecology approach. (ch1)
\begin{enumerate}
\item information is not easily stored on computers—and is not "data"
\item the more complex an information model, the less useful it will be
\item information can take on many meanings in an organization
\item technology is only one component of the information environment and often not the right way to create change.
\end{enumerate}

Davenport and Prusak are clear that describing a organization as an information ecology is a metaphor, but they think the value of the metaphor is as a counterweight to the machine-engineering point of view mentioned above.

In chapter 3 they write of four key attributes for information ecology

\begin{enumerate}
\item "(1) integration of diverse types of information;
\item (2) recognition of evolutionary change;
\item (3) emphasis on observation and description;
\item and (4) focus on people and information behavior.
\end{enumerate}
While the fully ecological approach to information would adopt all of these attributes, each is valuable in its own right. Adopting any of them will help organizations move in a more ecological direction—an important fact to keep in mind when information ecology seems like a daunting project."

In addition to describing the beliefs and attributes of their information ecology approach, Davenport and Prusak acknowledge that any organization is embedded in multiple information environments. The internal company or organizational environment is the main focus of their argument, but any organization is also connected to other organizations, such as business partners, customers, and suppliers, as well as being involved in a larger environment of the marketplace. The boundaries between these environments is often fuzzy and potentially quite porous. It is interesting to note that DP make no mention of the natural environment as an explicit context for an organization. This absence makes their overall program for adopting information ecologies less radical than it could potentially be.

The book is composed of multiple case studies, as is typical in the business management literature. Chapters in the book discuss the topic of strategy, politics, culture, staff, management, and architecture. Some of the cases are more intriguing than others. One case summarizes the information ecology approach into 5 uses for or understandings of information in an organization: truth, guidance, scarcity, accessibility, and weight. The only one of these which may be mapped easily onto biological conceptions of ecology is scarcity. Mapping the others to biological concepts would be a challenge.

\subsection{Nardi O'Day}