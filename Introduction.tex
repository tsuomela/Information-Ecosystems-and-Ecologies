\section{Introduction}

The use of the metaphor is arguably one of the more powerful learning devices that humans use to understand novel phenomena \cite{livingstone_1981, anderson_2016}. Consider the influence of Descartes' earth as a machine metaphor on scientific thinking over the last several centuries \cite{abram_1991} or perhaps the controversy that surrounds the suggestion that the earth is an auto-poetic organism from the Gaia Hypothesis \cite{lovelock_1974}. "Metaphors can play a crucial role in showing how the pieces [of the idea] fit together because they reveal the desired shape of the pattern [of the new concept] based on an idea that the listener already understands" \cite{anderson_2016}. In this context, an "information ecosystem" metaphor is widely used in academic libraries and has become nearly ubiquitous when speaking of the information systems that support scholarly communication and varied forms of data sharing and publication \cite[for example see]{walter_2008}. Sometimes alternate metaphors are used such as the "publishing ecosystem" or the "research ecosystem" \citep[for respective examples see][]{esposito_2013,dylla_2016}. The trending use of this language arises from non-academic contexts—for example in big data (the Hadoop ecosystem), email (the Thunderbird ecosystem), or industry conference titles (IEEE International Conference on Digital Ecosystems and Technologies)—and there remains little critical examination of either how this metaphor influences (open) data management practices within the academy or how these practices and the data themselves impact the natural environment \cite[although see][]{stepp_1999}. The use of this metaphor is also part of an even more widespread "everyday vernacular where it’s [the information ecosystem] about a large network of interconnected practices" that exist both within the information technology industry and the hard to describe information systems-society entanglement \cite{boyd_2016}. While it is likely an oversight in using readily available and understandable language to inform thinking on this novel entanglement, the definition of the ecosystem as a community of living organisms--plants, animals, fungi, and so on--and the set of relations with their surrounding non-living environment, is apparently not a part of the metaphor or the vernacular use of the term. Furthermore, it is possible that the current popularity of the "information ecosystem" metaphor is an ill-planned outcome of (un)intentional efforts in the 1990s to green the information economy.

The "information ecosystem" metaphor is both powerful and useful, nevertheless. The information systems--perhaps better named "data assemblages" \cite{kitchin_2014}, "digitally constructed memories" or "cross-domain knowledge bases" \cite{baker_2007}--that are being developed within institutions of research and education are like ecosystems in many ways. They contain diversity in human actors, institutions, norms and practices, and there is enourmous diversity within the data, metadata, and information itself. The systems are vast interconnected networks comprised of hierarchical levels in which technological and human actors play specific roles. There are serious questions about sustainability, albeit mostly economic, of the information system similar to those of natural ecosystems. What do we sustain, for who, and at what cost \cite[cf.][especially ch. 10]{liverman_2004, kitchin_2014}? There are limiting resources that govern the growth and evolution of the information systems including human and financial resources, material resources like spinning disks, the information itself, and perhaps even "raw data." We can observe dynamic equilibriums of stocks and flows in these information systems that are perhaps governed by the limiting factors above and ever-shifting interpretations of intellectual property law. Predator-prey relationships, such as those between publishers and authors, can be observed where lively competition takes place for the consumption and production of resources. In a similar vein thriving cooperative communities are built on shared needs for survival. Perhaps there are even emergent properties in the "information ecosystem" either akin to artificial intelligence arising from well formed linked open data, or simply a system so complex that no one person can understand its entirety. Indeed, the "information ecosystem" is a unknowable whole that is greater than the sum of its parts. It is evolving and growing in an almost lifelike fashion. Perhaps these assemblages and communities are even like an organisms.

While this list could go on, it is likely that we have stretched the metaphor beyond its useful limits (as one of the authors was recently scolded by Kevin Ashley, director of the IDCC); but this stretching is done in a purposeful manner. At a superficial level this paper interrogates the usefulness and limits of this metaphor by asking such questions as what are the similarities between an (open) data assemblage and an ecosystem? How can elemental cycles, flows of energy, and information exchange be conceptualized in the metaphor? How can the idea of contextualized communities, both human and non-human, be better expressed? And finally, are there shared characteristics of sustainable resource governance in ecosystem management and the management of (open) data? We also suggest a call for future research that returns to some of the original ideas of information ecology and human ecology - this direction benefits (traditional) ecology by including human aspects (technology and governance) and information ecology by including the natural environment. Through this explicit exploration of the ecology literature the paper provides an initial attempt to articulate the robust data literature within geography with the data literature from library and information science. Our hope is that this interdisciplinary engagement will contribute to the emergent fields of information and cognitive ecologies and provide more productive ways to understand novel roles of information and "open data" in socio-natural systems. Finally, we see this project as contributing to the many "ecologies" within the academy that attempt to bring social and natural sciences into productive dialog under the umbrella of the humanities.

The discussion begins with some comments on how to understand data and information. Then we review the history of ecological thought from a natural (traditional?) perspective to provide a foundation for interrogating the "information ecosystem" metaphor. This is followed by an exploration of the historical origins and initial uses of the metaphor within academic, industrial, and popular contexts. This light review of early work leads to a more thorough review of the major programmatic texts by Nardi and O'Day and Davenport and Prusak written in the late 1990s. Then the paper continues with a discussion of how ecological concepts have been borrowed in some other social scientific fields such as media studies, cognitive psychology, and political ecology. The next section of the paper discusses and comments on the usefulness of ecological metaphors in information science. [As I see it now we have two points to make. First is the critique portion showing how ecosystems talk in information science may be leading to incorrect or confusing ideas. Second is a constructive suggestion of how ecological ideas might be applied in a more rigorous manner to information studies, in other words how can our critique improve the discussion of information ecology in the future.] The analysis highlights the work that the metaphor does with particular attention given to how it might affect the environment. The paper concludes that the metaphor is useful and does constructive work, yet at the same time there are dangerous elements that render critical aspects of natural ecosystems invisible and tend to blur issues around data and social justice. [perhaps a call to speak of data assemblages as Kitchin suggests? Should we suggest an alternate term for complex information systems]
