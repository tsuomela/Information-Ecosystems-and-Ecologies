\section{Introduction}

The use of metaphors as tools is arguably one of the more powerful learning devices that we use to understand novel phenomena /cite{livingstone_1981}. In this context an "information ecosystem" metaphor is widely used in academic libraries and has become nearly ubiquitous when speaking of the information systems that support scholarly communication and varied forms of data sharing and publication. The trending use of this language arises from non-academic applications—for example in big data (the Hadoop ecosystem), email (the Thunderbird ecosystem), or industry conference titles (IEEE International Conference on Digital Ecosystems and Technologies)—and there remains little critical examination of either how this metaphor influences (open) data management practices within the academy or how these practices and the data themselves impact the natural environment \cite[although for contributions from human ecology see][]{stepp_1999}. The use of this metaphor is now part of an even more widespread "everyday vernacular where it’s [the information ecosystem] about a large network of interconnected practices" that exist both within the information technology industry and the hard to describe information systems-society entanglement \cite{boyd_2016}. Most likely a simple oversight in using readily available and understandable language to inform our thinking on this novel entaglement, the definition of the ecosystem as the set of relations between living organisms and their surrounding non-living environment is apparently not a part of the metaphor or the vernacular use of the term.

The "information ecosystem" metaphor is both powerful and useful, nevetheless. The information systems--perhaps better named digital libraries--that are being developed within institutions of reseach and education and broader research networks are like ecosystems in many ways. They contain diversity in human actors, institutions, norms and practices, and within the data, metadata, and information itself. The systems are vast interconnected networks comprised of heirarchical levels in which technological and human actors play specific roles. There are serious questions about sustainability, albeit mostly economic, of the information system similar to those of natural ecosystems. What do we sustain, for who, and at what cost \cite[cf.][especially ch 3(?)]{liverman_2004, kitchin_2014}? There are limiting resources that govern the growth and evolution of the information systems including human and financial resources, material resoures like spinning disks, the information itself, and perhaps even "raw data." We can observe dynamic equilibriums of stocks and flows in these information systems that are perhaps governed by the limiting factors above and ever-shifting interpretations of intellectual property law. Predator-prey relationships, such as those between publishers and authors, can be observed where lively competition takes place for the consumption and production of resources. In a similar vein thriving cooperative communities are built on shared needs. Perhaps there are even emergent properties in the "information ecosystem" either akin to artificial intelligence arising from well formed linked open data, or simply a system so complex that no one person can understand its entirety. Indeed the "information ecosystem" can be understood as a holon, a unknowable whole that is greater than the sum of its parts that is evoloving and growing in an almost lifelike fashion. 

While this list could go on, it is likely that we have strethed the metaphor beyond its useful limits.

At a superficial level this paper interrogates the usefulness and limits of this metaphor by asking such questions as what are the similarities between an (open) data assemblage \cite{kitchin_2014} and an ecosystem? How can elemental cycles and flows of energy be conceptualized in the metaphor? And finally, are there shared characteristics of sustainable resource governance in ecosystem management and the management of (open) data? As a part of a broader project, the paper provides an initial attempt to articulate the robust data literature within geograhy with the literature from library and information science. Our hope is that this interdisciplinary engagement will conribute to the emergent fields of information and cognitive egologies and provide more productive ways to understand novel roles of information (understood as "open data") in socio-natural systems. First we review the history of ecological thought from a natural perspective (traditional?). This is followed by an exploration of the historical use of the "information ecosystem" metaphor within academic, industrial, and popular contexts. and then explores how it is currently used within the academy. The analysis highlights the work that the metaphor does with particular attention given to how it might affect the environment. The paper concludes that the metaphor is useful and does constructive work, yet at the same time there are dangerous elements that render critical aspects of natural ecosystems invisible.
