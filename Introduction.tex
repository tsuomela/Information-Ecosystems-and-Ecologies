\section{Introduction}

Inspired by the phrase "information ecosystems" we begin with an historical-genealogical study of where the phrase came from. Inevitably there is a lot of overlap with the discipline of ecology, from which the concept of ecosystem has been borrowed. So in some sense ecology and ecosystem are treated as interchangeable. Perhaps this is something we need to develop in more detail.

The genealogical sections of the paper begin with some early examples of the phrase before the major programmatic texts by Nardi and ODay and Davenport and Prusak in the late 1990s. Then the paper continues with a discussion of how ecological concepts have been borrowed in some other social scientific fields such as media studies, cognitive psychology, and political ecology.

The next section of the paper discusses and comments on the usefulness of ecological metaphors in information science. As I see it now we have two points to make. First is the critique portion showing how ecosystems talk in information science may be leading to incorrect or confusing ideas. Second is a constructive suggestion of how ecological ideas might be applied in a more rigorous manner to information studies, in other words how can our critique improve the discussion of information ecology in the future.