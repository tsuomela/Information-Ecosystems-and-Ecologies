\section{Introduction}

The metaphor of the information ecosystem is widely used in academic libraries and has become nearly ubiquitous when speaking of the information systems that support scholarly communication and varied forms of data sharing and publication. The trending use of this language arises from non-academic applications—for example in big data (the Hadoop ecosystem) or email (the Thunderbird ecosystem)—and there remains little critical examination of either how this metaphor influences (open) data management practices within the academy or how these practices and the data themselves impact the natural environment. The use of this metaphor has arisen from an even more widespread "everyday vernacular where it’s [the information ecosystem] about a large network of interconnected practices" that exist both within the information technology industry and the hard to describe internet-society entanglement. Whether purposefully omitted or simply an oversight in using readily available and understandable language, the definition of the ecosystem as the set of relations between living organisms and their surrounding non-living environment is apparently not directly a part of the metaphor or the vernacular use of the term. 

This paper interrogates the usefulness and limits of this metaphor by asking such questions as what are the similarities between an (open) data assemblage and an ecosystem? How can elemental cycles and flows of energy be conceptualized in the metaphor? And finally, are there shared characteristics of sustainable resource governance in ecosystem management and the management of (open) data?  The paper describes the historical use of the information ecosystem metaphor and then explores how it is currently used within the academy. The analysis highlights the work that the metaphor does with particular attention given to how it might affect the environment. The paper concludes that the metaphor is useful and does constructive work, yet at the same time there are dangerous elements that render critical aspects of natural ecosystems invisible.
