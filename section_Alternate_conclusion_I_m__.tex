\section{Alternate conclusion}

[I'm putting this material in for eventual merger with the conclusion, this is a draft to try to explain the overall point of the paper in more direct terms which make sense to me as a non-ecological thinker.]

The word ecosystem functions in a complex way. At first glance it may suggest an interest in nature because of the presence of the eco- prefix and its common use as a term in the biological and environmental sciences. There is also a holistic implication that an ecosystem encompasses or contains the natural functions of a system. It is at this point when the concept of nature arises in our thoughts that the usage of the term ecosystem can become confused. The suggested presence of nature makes an ecosystem into a natural object, and this objectification makes the connection between a ecosystem and nature seem inevitable and given. When applied to human built systems, such as an economy, this implication of naturalism imbued in the very term being used begins to cloud and confuse the actual system being discussed. The problem with this confusion is that the very reason for using the term ecosystem in the first place, in order to emphasize the importance of nature and the non-human environment, is subtly elided and eliminated from the conversation. Nature disappears and we are left with  a human-built system, many times an economy, which may pay no more attention to the impacts of human activity on the environment as any other system we might have wished to condemn.

Consider the use of information ecology in the work of Davenport and Prusak. DP approach the issue as business management scholars who are trying to improve the human management processes for information within business organizations. From the beginning this is a human problem which only connects to nature at the most banal level of saying that all humans are part of nature. Human beings are part of nature, but that does not mean that everything human beings do is natural. This contrast is very difficult to explain clearly and applying the term ecosystem or ecology to human organizations and systems adds to the confusion.

Value judgments are made by the very use of the term - nature. To be natural is to be spontaneous, without processing or bias. But human organizations are artificial in the most basic sense - they are produced by human beings and maintained by human ingenuity and skill. They do not arise spontaneously, their organization is determined by rules, behavior, and hierarchies of human actions. In a sense they are fundamentally non-natural. To call these systems artificial is not derogatory, or at least it should not be consider that way. We should take pride in the artificial organizations we have created because they have accomplished some of the most complex tasks human beings are capable of, from forming governments to sending probes into outer space.

Acknowledging that these systems are artificial does not mean that the systems are in any way separated from nature. If ecology has taught us anything it is that human beings are deeply embedded in the natural world and cannot escape. Our organizations, systems, the vast quantities of information we have collected are artificial because we have built them. But they are built inside of an ecosystem, in  the most accurate use of the term, that includes human beings and everything else in the environment. Instead of referring to every collection of parts found in human civilization as an ecosystem we should refer to what they are artificial systems or economies which have been built by human beings for a specific purpose.

Prepending the word information to ecosystems does nothing to clarify the matter because the word information is almost as hard to define as nature. But there are many other uses of the term ecosystem that can be abandoned more quickly. When a newspaper story refers to the “music ecosystem” we should just prescribe that usage as laziness. What is really being referred to is the music economy or the music business, the complex of artificial organizations, individuals, and technology which human beings have built to distribute music.

When applying the term ecology or ecosystem to technology and information the same laziness is never far away. Phrases like “big data ecosystem” or “open source ecosystem” are more accurately referring to the economic, business, and technological structures which support these human activities.