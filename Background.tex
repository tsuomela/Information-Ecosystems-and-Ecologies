\section{Background: Data and Information}

Before we can discuss the nature of an "information ecosystem" in the context of libraries and research institutions, both the terms "data" and "information" must have context. We propose approaching these definitions with three perspectives. First, data and information can be considered separate but are directly related, yet the boundary between the two is often difficult to agree upon and the material world tends to underlie data while knowledge tends to emerge from information. The second perspective uses data levels to conceptualize different forms of data and information. Particularly within the context of academic research and open data, it is useful to differentiate between instrument data, processed data, analyzed data, and published data (perhaps an article with data included in the manuscript or with the data published as supplementary material). These two approaches are far from neat and overlap extensively. The third approach is more simple but much less precise. Everything that we perceive is data: texts are data, knowledge itself is data, audio and video are data, art is data, and so on. With this perspective data and information become almost the same. We expand on each of these approaches below.

To conceptualize the data/information relationship we first turn to the "knowledge pyramid" conceptualized as world > data > information > knowledge > wisdom. At the bottom of the pyramid is the world, and as data is gathered, processed, analyzed and understood, it is transformed and rises to higher but smaller levels on the pyramid. This is a useful model, but has several flaws. First, as described by Kitchin as an accident of history, the term data should perhaps be rethought. The Latin root \textit{datum} means that which is given, but in scientific research Nature does not give us data. Instead we as the research community actively take it \cite{kitchin_2014}. Furthermore, data is taken with purpose, it is never "objective" but instead is taken to answer a specific research question which automatically excludes other questions. Second the divisions between the named levels is far from clear. For example, the difference between data and information is largely subjective. Generally this difference lies in the amount of processing to transform and add value to the original data to make it into information. This difference quickly breaks down depending on the scale of analysis. Another common conception is that information carries meaning while data cannot. This difference breaks down quickly depending on who or what is interpreting meaning. These two examples might also apply to the differences between information and knowledge. Attention to these problems do not render the abstraction useless, but instead draws out weaknesses of which we need to be aware.

The National Aeronautics and Space Administration (NASA) uses a slightly different abstraction through the use of "data levels." In this model, developed as a part of the Earth Observing Data and Information System (EOSDIS), there are five data levels. Level zero, similar to the second step of the pyramid above, is the unprocessed instrument data (a sensor on a satellite). At level one the data has been processed and augmented with geospatial and temporal characteristics. Level two processing transforms the data into "geophysical variables" and level three data is mapped onto "uniform space-time grid scales." Outputs from models or derived results from lower level data is considered level four data \cite{nasa_2010}. Compared to the pyramid approach above, level two and level three data might be considered information and level four data might be considered knowledge. This imperfect mapping reminds us that both of these conceptual models for data are incomplete.

The final approach that we use is that \textit{everything} is data. If we examine data policies published by institutions of research and higher education this is often the approach we observe. For example: 

\begin{quote}
Research Data is defined as information recorded in physical form, regardless of form or the media on which it may be recorded. For the purposes of this policy, Research Data is further defined as including any records that would be used for the reconstruction and evaluation of reported or otherwise published results. Research Data also includes materials such as unmodified biological specimens, environmental samples, and equipment. Examples of Research Data and Materials include laboratory notebooks, notes of any type, photographs, films, digital images, original biological and environmental samples, protocols, numbers, graphs, charts, numerical raw experimental results, instrumental outputs from which Research Data can be derived and other deliverables under sponsored agreements. \cite{jhu_2008}
\end{quote}

According to the two previous abstractions outlined above, this definition of data includes information and knowledge from the knowledge pyramid and level four data from NASA's data levels. It is also likely that this example approximates the definition given to information in the "information ecosystem" metaphor used in the scholarly communication community. Information in this ecosystem includes data, published articles, and other artifacts from the scientific process.