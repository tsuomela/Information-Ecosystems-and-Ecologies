\section{Background: (Big) Data and Information}

Before we can discuss the nature of an "information ecosystem" in the context of libraries, research institutions and "open data", both the terms "data" and "information" must have context. We propose approaching "data" with three ontological perspectives in mind. First, data and information can be considered separate but are directly related. Additionally the boundary between the two is often difficult to agree upon. And finally the material world tends to underlie data and knowledge tends to emerge from information. The second ontology uses data levels to conceptualize different forms of data and information. Particularly within the context of academic research and open data, it is useful to differentiate between "raw" data, processed data, analyzed data, and published data (perhaps an article with data included either within the text or as an ). These two ontologies are far from neat and overlap extensively. The third ontology is more simple but much less precise: everything that we perceive is data. Using this approach texts are data, knowledge itself is data, audio and video are data, art is data, and so on. We expand on each of these ontological approaches below.

The well-used pyramidal progression conceptualized as world->data->information->knowledge->wisdom is a useful starting point but has several flaws. First, as described by Kitchin as an accident of history, the term data should perhaps be rethought. Data means that which is given, but in research Nature does not give us data. Instead we as the research community actively take it. Furthermore, data is taken with purpose, it is never "objective" but instead is taken to answer a specific question. Second the divisions between the named levels is far from clear. We find the difference between data and information to be largely subjective, although  . . . . . [there is a good cite on this] [more to come]

[Notes on data levels]