\section{Data and Information}

Before we can discuss the nature of an "information ecosystem" both the terms "data" and "information" must have context. We propose approaching definitions for these terms from three perspectives. 

First, data and information can be considered separate but are directly related; the boundary between the two is often difficult to agree upon and the material world tends to underlie data while knowledge tends to emerge from information. This perspective draws upon the "knowledge pyramid": world > data > information > knowledge > wisdom. At the bottom of the pyramid is the world, and as data is gathered, processed, analyzed and understood, it is transformed and rises to higher but smaller levels on the pyramid.

This is a useful model, but has several flaws. First the Latin root \textit{datum} means that which is given, but in scientific research Nature does not give us data, we actively take it \citep{kitchin_2014}. Thus data is never "objective" but instead is taken to answer a specific research question; this automatically excludes other questions. Second, the divisions between the pyramid levels is far from clear. For example, the difference between data and information is largely subjective. Another common conception is that information carries meaning while data cannot. This difference breaks down depending on who interprets meaning.

The second perspective uses "data levels" to conceptualize differences between data and information. Within the context of academic research it is useful to differentiate between instrument data, processed data, analyzed data, and published data. Compared to the pyramid approach above, processed and analyzed data might be considered information and published data might be considered knowledge.

This approach is comprable to the National Aeronautics and Space Administration (NASA) model in which there are five data levels: level zero is the unprocessed instrument data (a sensor on a satellite); level one data has been processed and augmented with geospatial and temporal characteristics; level two processing transforms the data into "geophysical variables"; and level three data is mapped onto "uniform space-time grid scales." Outputs from models or derived results from lower level data is considered level four data \citep{nasa_2010}. 

The third approach is more simple but much less precise. \textit{Everything} that we perceive is data: texts are data, knowledge itself is data, audio and video are data, art is data, and so on. If we examine data policies published by research institutions this is often the approach we observe. For example: 

\begin{quote}
Research Data is defined as information recorded in physical form ... including any records that would be used for the reconstruction and evaluation of reported or otherwise published results ... unmodified biological specimens, environmental samples, and equipment ... laboratory notebooks, notes of any type, photographs, films, digital images, original biological and environmental samples, protocols, numbers, graphs, charts, numerical raw experimental results ... \citep{jhu_2008}
\end{quote}

It is likely that this example approximates the definition given to information in the "information ecosystem" metaphor used in the scholarly communication community.
