\section{Background: (Big) Data and Information}

Before we can discuss the nature of an "information ecosystem" in the context of libraries, research institutions and "open data", both the terms "data" and "information" must have context. We propose approaching "data" from two perspectives to help the discussion that follows. First, data and information are directly related and the boundary between the two is often difficult to agree upon. Second, the material world tends to underlie data and knowledge tends to emerge from information. And third, when we think of data, particularly within the context of academic research and open data, it is useful to differentiate between "raw" data, processed data, analyzed data, and published data (perhaps called a text). 

The well-used pyramidal progression conceptualized as world->data->information->knowledge->wisdom is a useful starting point that needs several additional comments. First, described by Kitchin as an accident of history, the term data should perhaps be rethought. Data means that which is given, but in research Nature does not give us data, but instead we take it from her. Furthermore, data is taken with purpose, it is never "objective" but instead is taken to answer a specific question. Second the divisions between the named levels is far from clear. We find the difference between data and information to be largely subjective, although  . . . . . [there is a good cite on this] [more to come]

[Notes on data levels]