\section{Conclusion}

The "information ecosystem" metaphor is a powerful way to understand complexes of data, people, and machines in a rapidly changing social and technological environment. This metaphor has been used for nearly four decades now as a way to make sense of the "data revolution" of which we are just at the beginning. Furthermore, the field of information ecology has emerged through a partial co-evolution with the "information ecosystem" metaphor. This interdisciplinary field of study is producing valuable insights into how data and information mediate the human-environment relationship and is charting paths forward both at a philosophical and practical level. The work that is coming from the engagement with the long term ecological research network (LTER) and the field of information ecology seems particularly fruitful. There is also much to learn about how to engage with the human-data-environment relationships from other ecology-based sub disciplines such as political ecology and media ecology. 

For the most part those that use an information ecology approach do not use the "information ecosystem" metaphor, instead that metaphor is mostly found in the business management literature [is their enough evidence to say this yet, or should we suggest that more research needs to be undertaken to see if this is true]. Indeed, it seems that the use of the "information ecosystem" metaphor is often more of a colloquial use and it is often hard to discern exactly what work is being done by the metaphor [often it is imprecisely used]. Furthermore there is evidence that this usage emerged from attempts to make user interfaces for information analysis, curation and sharing systems more friendly and attractive for both academic and commercial uses. We suggest that terms such as "data assemblages" suggested by Kitchin, or "digitally constructed memories" and "cross-domain knowledge bases" \citep{baker_2007} are more appropriate reference to complexes of data, people, and machines than the "information ecosystem." We also suggest that the metaphor can be more literally used and that explicit engagements with the environmental outcomes of the data assemblages are possible much like work that uses the approach of political ecology. Information ecosystem work that explicitly examines these material outcomes and seeks practical solutions to mitigate negative environmental impacts is another beneficial way forward. [Google is doing some particularly interesting research and development in this context by cooling their server farms in Chicago with the icy winds that originate somewhere over Lake Michigan]

Within research contexts there is a strong managerial thread that follows from the initial throes of information ecology and information ecosystems which parallels that of (natural) ecosystems. Much ecology research was aimed at management prescriptions for conservation, restoration, and management of natural resources. This thread was quickly adopted by corporate business managers for the management of information resources, with the more efficient management of material resources as supply chains and commodities as an underlying goal. Somewhere along the way the management of (natural) ecosystems was lost. In the academy natural resource management endures as a purpose for (geographical) research and indeed two approaches have risen to dominate the agenda: political ecology and socio-ecological systems research. Both of these programs have as an overarching goal interdisciplinary engagements; both also engage with data and information as an elementary component of (natural) ecosystems. More research is needed that focuses on this shared interest in information in the ecosystem and how our all too human management of this data and information impacts environmental outcomes. While this call is reminiscent of cultural ecology of decades gone by, the recent insights of sustainability scholars such as Ostrom, legal scholars such as Boyle, and the scholars of the various ecology-based approaches suggest that a renewed engagement of this type will produce deeper knowledge of how the human-data-environment relationship functions at scale.

Finally, as the title of the paper suggests, information has long been recognized as a part fo the ecosystem. Indeed, information broadly conceived is what mediates the relationships between living organisms and their non-living environment. In the case of human beings, information that we gather from our surrounding environment is what we use to make decisions about how we govern and use natural resources. Information has always been and will always be a part of this ecosystem, but to conceive of an information ecosystem that is separate from our relationship with nature may not be in our best interest as but one species that makes a home on planet earth. While metaphors may be as useful way to understand novel phenomena in scientific settings, their forms and roles within learning processes are still under debate within the philosophy of science literature. Like Bohr's planetary model of the atom, some metaphors are destined for retirement as our understandings of the phenomena deepen. Perhaps it is time to retire the information ecosystem metaphor as well as we begin to better understand the data revolution.

[something on value??? yikes!]