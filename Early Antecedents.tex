\section{Early Antecedents}

The idea that the flow of ideas and information is part of ecosystem processes can be traced back at least one-hundred years to the work of Verdansky, Le Roy and Teilhard in the 1920s. These thinkers proposed that human thought, ideas, and creativity--in short human cognition--transforms the biosphere through purposeful manipulative processes at the atomic level. Much like early life had transformed the geosphere into the biosphere, the development of human ideas would transform the biosphere into the \textit{noosphere}: literally translated from its Greek roots, the sphere of the mind. While the noosphere never gained much traction, many ecologists and geologists now call the current geological epoch the \textit{anthropocene} as a purposeful admission that human ideas, technology, and governance have tangible and observable effects on the global environment. 

Another early link between ecology and information science appeared at approximately the same time as the appearance of the word "ecosystem". The librarian Ranganathan famously suggested five laws of librarianship of which the last of his laws is that the Library is a growing organism. This metaphor, like other organismic metaphors popular at this time, included notions of growth, evolution, a touch of diversity, and some vitality. The library is literally alive. \citep{ranganathan_1931}. While these two examples may appear far-fetched and dated, their influence is remains far-reaching. 

A good example comes from human ecology and cultural ecology in the field of anthropology. Similar to how ecologists measured flows of energy and nutrients during the quantitative revolution, anthropologists counted calories and food exchanges in small "closed" societies. Claims were made that culturally defined norms, rules, and rituals govern homeostatic socio-natural systems. The flow and management of information controlled by cultural institutions provides the feedback loops to maintain system stability. The anthropologists borrowed from systems theory, cybernetics, and ecology to make their arguments. 

Categorizing socio-natural systems as homeostatic and self-regulating rose in popularity through the 1960s and 1970s only to be debunked as the scale of analysis expanded (see human ecology in figure 2). Bernard Nietschmann, a cultural anthropologist who studied coastal Miskito communities in Nicaragua, clearly showed that if the scale of analysis includes international trade pressures, the expansion of capitalism as an economic system, and other external factors, stability of these "closed" systems quickly morphs into cultural and ecological degradation \citep{nietschmannn_1973}.

The idea that information is a key component of an ecosystem remains nevertheless. Information influences the formation of biosystems, the health of human beings, and our social well-being. All of these depend upon systems for valuing, storing, transmitting, and receiving information, be it digital or analog. Similar to the human ecology, a program of informmation ecology has the potential to be a grand synthesis between the biological sciences and the human condition through the careful study of information exchange \citep{eryomin_information_1998}.\footnote{This recurring theme has again been picked up very recently as an extension of the adaptive management (AM) and Panarchy literatures \citep{eddy_information_2014}} 

A common thread in these early examples is a shared basis in the management of resources; elemental resources, information resources, and natural resources. This management emphasis was picked up in business management literature and emerged as a program of information ecology in the 1990s. 

Horton Forest is perhaps the first author to use the term "information ecology" in an article title in his 1978 publication on computer systems management \citep{forest_1978}. Kevin Harris provides one of the clearest early examples from the business management literature in his short article titled "Information Ecology" in which he recognizes the dynamic interdependence of information systems within an organization. He analogizes the interaction between systems within an organization as a type of ecosystem in the sense that they are self-regulating, progressive, and maintain states of dynamic equilibrium \citep{harris_information_1989}. 