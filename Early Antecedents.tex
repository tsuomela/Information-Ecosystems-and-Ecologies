\section{Early Antecedents}

The idea that the flow of information is a part of ecosystem processes can be traced back at least one-hundred years to the work of Verdansky, Le Roy and Teilhard in the 1920s. Preceding the rise of ecosystem thinking by perhaps a decade these thinkers proposed that the human thought, ideas, and creativity--in short human cognition--would give rise to the human power over nuclear structures in atoms so that people might literally create new elements. Much like life had transformed the geosphere into the biosphere, the development of human ideas would transform the biosphere into the \textit{noosphere}: literally translated from its Greek roots, the sphere of the mind. While this particular outcome has not yet become manifest as originally conceived, many ecologists and geologists now call the current geological epoch the \textit{anthropocene}. Naming the current geological epoch as such is an purposeful admission that human ideas, technology, and governance have tangible and observable effects on the global environment, albeit through alternative means from Verdansky. 

Another early link between ecology and information science appeared at approximately the same time as the appearance of the word "ecosystem". The librarian Ranganathan famously suggested five laws of librarianship of which the last of his laws is that the Library is a growing organism. This metaphor, like other organismic metaphors popular at this time, included notions of growth, evolution, a touch of diversity, and some vitality. The library is literally alive (based on a unidentifiable Sanskrit religious metaphor - the library is like a god[???]) \citep{ranganathan_1931}.

While these two examples may appear far-fetched and dated, the influence of these metaphors is far-reaching. Perhaps the best examples come from the development of human ecology and cultural ecology in the field of anthropology. Similar to how ecologists measured flows of energy and nutrients to assuage their lack of scientific and statistical self-confidence during the quantitative revolution, anthropologists counted calories and food exchanges in small "closed" societies. Claims were made that culturally defined norms, rules, and rituals govern homeostatic socio-natural systems. The flow and management of information controlled by cultural institutions in these systems provided the feedback loops to maintain system stability. Much like ecologists, the anthropologists were borrowing from systems theory and cybernetics to make their arguments. This kind of thinking rose in popularity through the 1960s and 1970s only to be debunked as the scale of analysis expanded. Bernard Nietschmann, a cultural anthropologist who studied coastal Miskito communities in Nicaragua, clearly showed that if the scale of analysis includes international trade pressures, the expansion of capitalism as an economic system, and other external factors, stability of these "closed" systems quickly morphs into cultural and ecological degradation \citep{nietschmannn_1973}.

A common thread in these three examples is that all have a basis in the management of some kind of resource; Verdansky's noosphere in elemental resources, Ranganathan's living library in information resources, and Nietschmann's socio-ecological environments in turtles as biological resources. This emphasis on resource management was picked up in business management literature and emerged as the management of the information ecosystem, or information ecology. At first these terms appeared in disconnected places. Horton Forest is perhaps the first author to use the term "information ecology" in an article title in his 1978 publication on computer systems management \citep{forest_1978}. Nelson's 1985 book titled \textit{An Evolutionary Theory of Economic Change} suggested biological analogies in economic theory \citep{nelson_evolutionary_1985}. This work was later used as a base for some of the metaphors in Davenport's work [I want to cut this, there are many other examples of this as well "The Economy as an Evolving Complex System" Anderson, Arrow, and Pines, 1988 for example (from economic and complexity theory), another is "State, Law, and Economy as Autopoietic Systems" by Teubner and Febbrajo, 1992]. Kevin Harris provides one of the clearest early examples from the business management literature in his short article titled "Information Ecology" where he calls upon researchers to recognize the dynamic interdependence of information systems within an organization, especially businesses. According to him, organizations have a state of informedness shaped by the exchange of information from both inside and outside of an organization. He analogizes this interaction between systems within an organization as a type of ecosystem in the sense that they are self-regulating, progressive, and maintain states of dynamic equilibrium [what do you think this means to him? Stable or chaotic? autopoiesis in organizations]. Interestingly, he likens the goals of information managers to maintain "dynamic but stable" systems to the processes found in naturally occurring ecosystems, he even states that moderate levels of "disturbance" may be healthy in an information ecosystem. His authority for these claims is a 1968 study in the social sciences that draws the same conclusion: social systems maintain dynamic equilibriums [again, how is this used? written before May's article on chaos in ecology] much like those found in naturally occurring ecosystems. [He also has a background as a fellow with the British Library and at the time of writing his article was managing databases for a tech firm in London.] In any case the discussion is brief and not elaborated in any detail, but it is suggested that it may be a "useful analogy" \citep{harris_information_1989}.

Another interesting precursor use of the phrase "information ecologies" comes from the environmental studies literature. Eryomin adduces the importance of information to the study and understanding of the environment. Information influences the formation of biosystems, the health of human beings, and our social well-being. All of these depend upon systems for valuing, storing, transmitting, and receiving information, be it digital or analog. A program for the study of information ecology would examine each of these processes, in addition to examining the criteria for information and the communities and organizations through which it flows. For Eryomin, information ecology, similar to the earlier human ecology, has the potential to be a grand synthesis between the biological sciences and the human condition through the careful study of information exchange \citep{eryomin_information_1998}. This recurring theme has again been picked up very recently as an extension of the adaptive management (AM) and Panarchy literatures \citep{eddy_information_2014}.