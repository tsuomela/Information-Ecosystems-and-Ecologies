Finding a coherent program for information ecology before the publication of \cite{nardi_information_1999, davenport_information_1997} is challenging. When the term was used it appeared in disconnected places. Harris is one of the earliest examples from the business management literature. His short note on information ecology called upon researchers to recognize the dynamic interdependence of information systems within an organization, especially businesses. According to him, organizations have a state of informedness shaped by the exchange of information from both inside and outside of an organization. He analogizes this interaction between systems within an organization as a type of ecosystem. The discussion is brief and not elaborated in any detail \cite{harris_information_1989}.

The connection to the business and management literatures suggests another potential connection to evolutionary economics. "An Evolutionary Theory of Economic Change" \cite{nelson_evolutionary_1985} is one example of ideas from the biological sciences being imported into economics. The latter work by Davenport and Prusak is perhaps the culmination of and concresence of evolutionary economics and organizational science, with its concern over management information systems.

Another interesting precursor use of the phrase "information ecologies" comes from the environmental studies literature. Eryomin adduces the importance of information to the study and understanding of the environment. Information influences the formation of biosystems, the health of human beings, and our social well-being. All of these depend upon systems for valuing, storing, transmitting, and receiving information. A program for the study of information ecology would examine each of these processes, in addition to examining the criteria for information and the communities and organizations through which it flows. For Eryomin, information ecology has the potential to be a grand synthesis between the biological sciences and the human condition through the careful study of information exchange \cite{eryomin_information_1998}